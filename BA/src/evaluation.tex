%% eval.tex
%% $Id: eval.tex 5 2005-10-10 20:55:48Z bless $

\chapter{Evaluation}
\label{ch:Evaluation}
Die Evaluation der Ergebnisse beginnt damit, die Daten der eSense Earpods in Fenster (\textit{windows}) einzuteilen.
Auf diese Fenster werden anschließend \textit{Features} berechnet, welche ausschlaggebend für die Klassifizierung sein sollen.

\section{Gibt es passende Features?}
Die Suche nach passenden Features wurde mit dem Python Package \texttt{tsfresh} angegangen.
\texttt{tsfresh} berechnet automatisch Charakteristiken anhand von einer \textit{Timeseries}, die sogenannten Features \todo{ref to website https://tsfresh.readthedocs.io}.
Zudem evaluiert das Package die Relevanz der Charakteristiken, um somit die Regression oder die Klassifikation zu erleichtern.


passende features sind z.B
\begin{itemize}
    \item gyroZ partial autocorrelation
    \item gytoX fft coefficient
    \item gyroX agg autocorrelation
    \item accY autocorrelation
    \item gyroY change quantiles 
    \item gyroZ fft coefficient
\end{itemize}


\section{Vergleich der Klassifikationsverfahren}
\begin{itemize}
    \item SVM
    \item Random Forest
    \item XGBoost
\end{itemize}

Was bieten die verfahren, wie macht das sinn, dass sinnvolle ergebnisse herauskommen...

WelcheVor-undNachteilekönnendieseVerfahrenbieten?


\section{Ablauf der Evaluierung}
Im Kapitel \ref{ch:Implementierung:classification_pipeline} wurde beschrieben, wie die Features berechnet und persistiert wurden. 

\begin{itemize}
    \item in welche windows wird das ganze aufgeteilt? 5 sec, 10sec?
    \item welche zeischlitze werden gelabelt, welche nicht?
    \begin{itemize}
        \item jeder wert in den 5-10 sec windows ist gelabelt, ab wann wird window gelabelt? 50\%, 90\%?
        \item sollen windows weggelassen werden, bei denen das verhältnis zwischen 10 und 90\% ist? damit die Übergänge nicht die ergebnisse verfälschen? WIR SAGEN NEIN, SONST KANNS DANACH NICHT VERWENDET WERDEN
    \end{itemize}
\end{itemize}
ok, jz sind die features der daten persistiert, jz wird klassifiziert

für jede classification
\begin{itemize}
    \item Leave one Subject out, dann Mittelwert von jedem, was rausgelassen wurde
    \item leave one Subject out, jedoch von jeder Position einzeln, dann Mittelwert von jedem loso.
\end{itemize}

\subsection{Ergebnisse}
\todo{plots hier rein}

\section{Erkenntnisse}
\todo{ich habe gelernt}
\begin{itemize}
    \item auswertungsdaten: Auf dem rücken liegende ddaten sind am vielversprechendsten
    \item die atempausen können einigermaßen klassifiziert werden
    \item rauschen entfernen bringt nix vermutlich wegen tsfresh, da es das schon macht, also es kommen
    \item 
\end{itemize}


gibt earables, die blutsauerstoff und puls mittracken können
plus grafik telegram tobi