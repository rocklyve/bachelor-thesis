%% eval.tex
%% $Id: eval.tex 5 2005-10-10 20:55:48Z bless $

\chapter{Evaluation}
\label{ch:Evaluation}

\section{Vergleich verschiedener Klassifikationsverfahren}
\begin{itemize}
    \item SVM
    \item Random Forest
    \item XGBoost
\end{itemize}

Was bieten die verfahren, wie macht das sinn, dass sinnvolle ergebnisse herauskommen...

WelcheVor-undNachteilekönnendieseVerfahrenbieten?

\section{Gibt es passende Features?}
\begin{itemize}
    \item tsfresh verwendet
    \item es hat sich ein pool an resultaten herausgezeichnet, war oft n 
    \begin{itemize}
        \item gyroZ partial autocorrelation
        \item gytoX fft coefficient
        \item gyroX agg autocorrelation
        \item accY autocorrelation
        \item gyroY change quantiles 
        \item gyroZ fft coefficient
    \end{itemize}
\end{itemize}

\section{Ablauf der Evaluierung}
Im Kapitel \ref{ch:Implementierung:classification_pipeline} wurde beschrieben, wie die Features berechnet und persistiert wurden. 

\begin{itemize}
    \item in welche windows wird das ganze aufgeteilt? 5 sec, 10sec?
    \item welche zeischlitze werden gelabelt, welche nicht?
    \begin{itemize}
        \item jeder wert in den 5-10 sec windows ist gelabelt, ab wann wird window gelabelt? 50\%, 90\%?
        \item sollen windows weggelassen werden, bei denen das verhältnis zwischen 10 und 90\% ist? damit die Übergänge nicht die ergebnisse verfälschen? WIR SAGEN NEIN, SONST KANNS DANACH NICHT VERWENDET WERDEN
    \end{itemize}
\end{itemize}
ok, jz sind die features der daten persistiert, jz wird klassifiziert

für jede classification
\begin{itemize}
    \item Leave one Subject out, dann Mittelwert von jedem, was rausgelassen wurde
    \item leave one Subject out, jedoch von jeder Position einzeln, dann Mittelwert von jedem loso.
\end{itemize}

\subsection{Ergebnisse}
\todo{plots hier rein}

\section{Erkenntnisse}
\todo{ich habe gelernt}
\begin{itemize}
    \item auswertungsdaten: Auf dem rücken liegende ddaten sind am vielversprechendsten
    \item die atempausen können einigermaßen klassifiziert werden
    \item rauschen entfernen bringt nix vermutlich wegen tsfresh, da es das schon macht, also es kommen
    \item 
\end{itemize}


gibt earables, die blutsauerstoff und puls mittracken können
plus grafik telegram tobi