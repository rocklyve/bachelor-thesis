%% grundlagen.tex
%% $Id: grundlagen.tex 28 2007-01-18 16:31:32Z bless $
%%

\chapter{Basics \& Related Work}
\label{ch:Basics}

\section{Schlafmedizin}
\label{ch:Basics:se:schlafmedizin}
% start seite 3-5 im buch schlafmedizin_1x1 
Unter dem Begriff \textit{Schlafmedizin} versteht man laut \cite{schlafmedizin_1x1} die Lehre von Diagnostik, Klassifikation und Behandlung von Störungen währrend des Schlafs. 
Trotz Erwähnungen in der Antike findet Schlafmedizin erst seit des letzten Jahrhunderts Bedeutung.
Mithilfe der Polysomnographie konnten unterschiedliche Schlafphasen zyklischen Ablaufs erkannt werden.
Zudem konnten den Schlafphasen physiologische Eigenschaften nachgewiesen werden.
Die Polysomnographie misst Gehirnströme, Augenbewegungen und Muskelspannungen.
Mithilfe dieser Informationen konnte man Schlafkrankheiten erkennen und mit der Behandlung dieser beginnen.


Heutzutage sind ca. 80 Schlafstörungen in dieversen Bereichen bekannt, welche neben psychologischen Testverfahren überwiegend elektrophysiologisch behandelt/ untersucht werden.
Patienten werden mit ambulanten Hilfsmitteln oder stationär in einem Schlaflabor untersucht und anschließend von einem technisch ausgebildeten Personal analysiert.
Im Falle eines Schlaflabors, welches genauere Messergebnisse im Vergleich zu einem ambulanten Hilfsmittel (z.B eine Langzeitbewegungsmessung), wird eine Polysomnographie durchgeführt.
\todo{dieser vorhergehende Satz kann evtl weg.}
% ende cite
% start seite 9- im buch schlafmedizin_1x1 
Im Schlaflabor werden neben elektrophysiologischen Messungen des Schlafs auch Untersuchungen der Müdigkeit, der Tagesschläfrigkeit und der Aufmerksamkeit \cite{schlafmedizin_1x1} vorgenommen.
% ende cite

\todo{Seite 13,14 in 1x1, schlaf erklärt, wie psg was klassifiziert usw... wichtig!!!}

\section{Schlaf: Respiratorische Ereignisse}

\todo{Seite 81+ 1x1... anschauen, erklären, was respiratorische Ereignisse sind}

Welche respiratorischen Ereignisse gibt es \& wie unterscheiden sie sich (Apnoe, Hypopnoe, Hyperventilation, ...)?
\todo{Apnoes unterscheiden und erklären}

\subsection{Zentrales Schlafapnoe}
\todo{Zentrales apnoe im 1x1 auf 86 erklärt...}
\todo{Schlaf von Apnoepatienten 91 1x1}
\todo{126-127 in praxis der schlaf...}

\todo{Diagnostik 95 1x1}

\section{Klassifizierung von Schlafstörungen}
Nach \cite{praxis_der_schlafmedizin} können Schlafstörungen durch die Möglichkeit, viele Biosignale währrend des Schlafes zu registrieren, genauer charaktisiert werden.
Basierend auf dieser Annahme wurden folgende Klassifikationen für Schlafstörungen entwickelt.
\begin{itemize}
    \item Ein- und Durchschlafstörungen (Insomnien)
    \item Schlafbezogene Atmungsstörungen
    \item Hypersomnien zentralnervösen Ursprungs
    \item Zirkadiane Rhythmusschlafwachstörungen
    \item Störungen in Verbindung mit Schlaf, Schlafstadien oder partiellem Erwachen (Parasomnien)
    \item Schlafbezogene Bewegungsstörungen
    \item Andere Schlafstörungen
\end{itemize}

Diese Gliederung ist in \cite{praxis_der_schlafmedizin} zu finden und orientiert sich an der \textit{ICSD-3}.

Welche Alternativen gibt es zur Aufzeichnung und Klassifikation?

\section{Forschung von Klassifikation anhand von IMU-Daten}
Welche Forschung gibt es zur Klassifikation von respiratorischen Ereignissen mit IMUs