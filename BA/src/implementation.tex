\chapter{Implementierung}
\label{ch:Implementierung}

In diesem Kapitel wird erläutert, wie die Daten des Datensatzes gesammelt, zur weiteren Verarbeitung vorbereitet und schließlich analysiert werden.
Der Fokus hierbei liegt auf der Implementierung, für mehr Details, siehe Kapitel \todo{link to ref} %verlinke auf kapitel, wo erklärt wird, was wie gemacht wird

\section{App}
\subsection{Plattform}
Die Smartphone-App wurde mit der Sprache Swift für Apple-Smartphones entwickelt. 
Mit der Software XCode lässt sich eine mit Swift geschriebe App kompilieren und auf dem Smarphone installieren.
\todo{insert table of frameworks}

Zur Einbindung externer Frameworks wird der Dependency-Manager \textit{Accio} und \textit{Carthage} verwendet.
Folgende Frameworks sind in der App eingebunden worden:
\todo{insert table of frameworks with description}

Durch das Framework \textit{Imperio} ist es möglich,
\todo{erklaere imperio pattern?}

\subsection{Messungsaufbau}
App Implementierung zur messung von esense daten wie gemacht?

manuelle ble connection, daten werden manuell herausgezogen, in Datenbank persistiert... 

\todo{extend with diagrams like class-diagram}

\todo{screenshots of app}

\section{Anbindung an Auswertungspipeline}
Zum aktuellen Stand liegen die Daten der eSense-Earpods und vom PSG-System vor. 
Zu Beginn müssen die PSG-Daten, welche als eine Messung für alle 3 Positionen pro Studienteilnehmer persistiert wurde, in 3 einzelne Messungen aufgeteilt werden.
Die Daten des PSG-Systems liegen als \textit{edf}-Datei vor. 
Diese können mittels python und der Library \textit{edfrd} (siehe \todo{verlinke zu tools})

Wie werden Daten ausgelesen und in richtige Form gebracht

von app exportiert als zip, danach verarbeitung erklaeren

\begin{itemize}
    \item asdf
\end{itemize}

\subsection{Synchronisation der Daten}
wie wird sichergestellt, dass die Zeitmessung synchron ist von eSense zu PSG

\section{Verarbeitungspipeline zur Klassifikation}
Wie werden Daten aufgeteilt,wie wird trainiert?
