%% LaTeX-Beamer template for KIT design
%% by Erik Burger, Christian Hammer
%% title picture by Klaus Krogmann
%%
%% version 2.1
%%
%% mostly compatible to KIT corporate design v2.0
%% http://intranet.kit.edu/gestaltungsrichtlinien.php
%%
%% Problems, bugs and comments to
%% burger@kit.edu

\documentclass[18pt]{beamer}

%% SLIDE FORMAT

% use 'beamerthemekit' for standard 4:3 ratio
% for widescreen slides (16:9), use 'beamerthemekitwide'
% for widescreen slide without sidebar use 'beamerthemekitwidenosidebar'

\usepackage{templates/beamerthemekit}
%\usepackage{templates/beamerthemekitwide}
%\usepackage{templates/beamerthemekitwidenosidebar}

% use this to disable the latex beamer navigation symbols
%\beamertemplatenavigationsymbolsempty


%% TITLE PICTURE

% if a custom picture is to be used on the title page, copy it into the 'logos'
% directory, in the line below, replace 'mypicture' with the 
% filename (without extension) and uncomment the following line
% (picture proportions: 63 : 20 for standard, 169 : 40 for wide
% *.eps format if you use latex+dvips+ps2pdf, 
% *.jpg/*.png/*.pdf if you use pdflatex)

%\titleimage{mypicture}

%% TITLE LOGO

% for a custom logo on the front page, copy your file into the 'logos'
% directory, insert the filename in the line below and uncomment it

\titlelogo{Logo_TECO}

% (*.eps format if you use latex+dvips+ps2pdf,
% *.jpg/*.png/*.pdf if you use pdflatex)

%% TikZ INTEGRATION

% use these packages for PCM symbols and UML classes
% \usepackage{templates/tikzkit}
% \usepackage{templates/tikzuml}

% the presentation starts here

\title[Proposal Bachelorarbeit]{Proposal zur Bachelorarbeit:\\ Titel der Bachelorarbeit}
\subtitle{Professor: Michael Beigl, Betreuer: Tobias Röddiger}
\author{David Laubenstein}

\institute{Lehrstuhl Pervasive Computing Systems}

% Bibliography

\usepackage[citestyle=authoryear,bibstyle=numeric,hyperref,backend=biber]{biblatex}
\addbibresource{templates/example.bib}
\bibhang1em

\begin{document}

% change the following line to "ngerman" for German style date and logos
\selectlanguage{ngerman}

%title page
\begin{frame}
\titlepage
\end{frame}

% --------------------------- 1. Folie PIBA ----------------------------------

\section{PIBA}
\subsection{Problem}
\begin{frame}{Problem}
Problem: Was ist das Problem? (allgemeinverständlich für Informatiker, nach Möglichkeit graphisch)
Idee: Was ist ihr Lösungsansatz?
Vorteile (Benefit): Was ist der Vorteil ihres Lösungsansatzes? (auch ggü. anderen Ansätzen)
Aktionen (Action): Welche Schritte wollen Sie gehen?
\end{frame}

\subsection{Idee}
\begin{frame}{Idee}
Problem: Was ist das Problem? (allgemeinverständlich für Informatiker, nach Möglichkeit graphisch)
Idee: Was ist ihr Lösungsansatz?
Vorteile (Benefit): Was ist der Vorteil ihres Lösungsansatzes? (auch ggü. anderen Ansätzen)
Aktionen (Action): Welche Schritte wollen Sie gehen?
\end{frame}

\subsection{Benefit}
\begin{frame}{Benefit}
Problem: Was ist das Problem? (allgemeinverständlich für Informatiker, nach Möglichkeit graphisch)
Idee: Was ist ihr Lösungsansatz?
Vorteile (Benefit): Was ist der Vorteil ihres Lösungsansatzes? (auch ggü. anderen Ansätzen)
Aktionen (Action): Welche Schritte wollen Sie gehen?
\end{frame}

\subsection{Action}
\begin{frame}{Action}
Problem: Was ist das Problem? (allgemeinverständlich für Informatiker, nach Möglichkeit graphisch)
Idee: Was ist ihr Lösungsansatz?
Vorteile (Benefit): Was ist der Vorteil ihres Lösungsansatzes? (auch ggü. anderen Ansätzen)
Aktionen (Action): Welche Schritte wollen Sie gehen?
\end{frame}

% ----------------------- 2. Folie Umsetzung ihres Lösungsansatzes -------------
\section{Umsetzung ihres Lösungsansatzes}
\begin{frame}{Umsetzung ihres Lösungsansatzes}
\begin{exampleblock}{Example 1}
\begin{itemize}
\item Bullet point 1
\item Bullet point 2
\item \dots
\end{itemize}
\end{exampleblock}
\end{frame}


% ----------------------- 3. Folie Geplante Evaluation -------------
\section{Geplante Evaluation}
\begin{frame}{Geplante Evaluation}
\begin{exampleblock}{Example 1}
Geplante Evaluation (Form, Ziel, Metriken, Fragestellungen, ggf. Hypothesen)
\end{exampleblock}
\end{frame}

% ----------------------- 4. Folie Zusammenfassung -------------
\section{Zusammenfassung}
\begin{frame}{Zusammenfassung}
\begin{exampleblock}{Example 1}
4. Folie: Zusammenfassung, dazu einfach Folie 1 nochmals auflegen (Übersicht; gibt auch bei der anschließenden Diskussion nochmal einen Überblick
\end{exampleblock}
\end{frame}

\appendix
\beginbackup

\begin{frame}[allowframebreaks]{References}
\printbibliography
\end{frame}

\backupend

\end{document}
